% Created 2022-08-13 Sat 01:33
% Intended LaTeX compiler: pdflatex
\documentclass[a4paper]{article}
\usepackage[utf8]{inputenc}
\usepackage[T1]{fontenc}
\usepackage{graphicx}
\usepackage{longtable}
\usepackage{wrapfig}
\usepackage{rotating}
\usepackage[normalem]{ulem}
\usepackage{amsmath}
\usepackage{amssymb}
\usepackage{capt-of}
\usepackage{hyperref}
\usepackage[top=1cm,left=3cm,right=3cm]{geometry}
\author{Jeetaditya Chatterjee}
\date{\today}
\title{Flymake}
\hypersetup{
 pdfauthor={Jeetaditya Chatterjee},
 pdftitle={Flymake},
 pdfkeywords={},
 pdfsubject={},
 pdfcreator={Emacs 29.0.50 (Org mode 9.6)}, 
 pdflang={English}}
\begin{document}

\maketitle

\section*{What is flymake?}
\label{sec:org423a754}
\begin{notes}
Good place to start. flymake is a generic package that provides diagnostics in a
buffer. its included in emacs by default and is a drop in replacement for the
main diagnostic package we all use, flycheck.
\end{notes}
\subsection*{Ok\ldots{} whats a diagnostic?}
\label{sec:org86dea48}
\begin{notes}
Now this may seem a little tedious but its worth going over. in short a
Diagnostic is a piece of information tied to a part of your text. The main
example of this is a syntax error. displayed through your linter. but it can be
anything. From style guides, to spelling mistakes to your managers nagging
encoded into a command line application.
\end{notes}
\section*{Why replace flycheck then?}
\label{sec:orgc997eda}
\begin{notes}
Well\ldots{} in reality if flycheck works then there can be little reason to switch.
that being said some of the reasons to switch include
\begin{itemize}
\item Its built in
That means there is less to install and it will be maintained for the
forceable future.
\item It composes nicely
what I mean by that is that instead of reinventing the wheel and creating its
own system of activation. Instead of specially registering modes to diagnostic
checkers. you add your ``flymake backends'' (a function) to
\texttt{flymake-diagnostic-functions}. If you want to make it mode
\item It basically does not need to be configured
For this point we shall channel our inner taco and learn to love the defaults.
I kid but its basically true, for the most part this is my config
\end{itemize}
\end{notes}

\begin{itemize}
\item Its built in
\item It ``composes'' nicely
\item It basically does not need to be configured
\end{itemize}
\begin{notes}
All of this should be pretty understandable. And for the most part that's it. if
you use lsp-mode it should all work pretty magically. and because diagnostics
are pretty simple in there nature most people can do this without obstructing
there workflow!
\end{notes}
\begin{verbatim}
;; Activate ~flymake-mode~ by default
(add-hook! (prog-mode text-mode) #'flymake-mode)

;; Tell lsp-mode we /really/ want to use ~flymake~
(after! lsp-mode
  (setq lsp-diagnostics-provider :flymake))

;; put the indicators in the right (->) fringe
(after! flymake
  (setq flymake-fringe-indicator-position 'right-fringe))
\end{verbatim}
\end{document}
