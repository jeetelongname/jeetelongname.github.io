\documentclass[notes]{beamer}
% \documentclass[notes=only]{beamer}

\usepackage{minted}
\usepackage[style=apa, backend=biber]{biblatex}
\addbibresource{latex.bib}
\usetheme{default}
\usecolortheme{seahorse}

\title{A small introduction to \LaTeX{}}
\subtitle{}
\author{Jeetaditya Chatterjee (Jeet)}
\institute{UoP IT Society}
\date{}
\logo{\includegraphics[width=2cm]{itsoc.png}}

\begin{document}

\begin{frame}
	\titlepage{}
\end{frame}
\begin{frame}{What we will cover}
	\tableofcontents
\end{frame}
\AtBeginSection
{
	\begin{frame}{Where are we?}
		\tableofcontents[currentsection]
	\end{frame}
}

\section{Why Stray From The Known?}%
\label{sec:whystray}
\begin{frame}{What are the problems with the current form?}
	\begin{itemize}
		\item Formatting is hard
		\item Automatic is better than manual
		\item Explicit is better than Implict
		\item Declarative is better than Speculative
	\end{itemize}
\end{frame}

\begin{frame}{Why Stray from the Known?}
	\begin{itemize}
		\item \LaTeX{} is designed to produce large technical documents
		\item You can format once at the beginning and then focus on writing
		\item You can break up large documents without having too worry about
		      formatting
		\item \LaTeX{} has been around forever and you can find a package or
		      use case for most document preparation needs
		\item Its the standard for academic and scientific writings, So much
		      so the school of computing blesses its use for larger documents
	\end{itemize}
\end{frame}

\begin{frame}{How to get started with LaTeX}
	\begin{center}
		\url{https://www.overleaf.com}
	\end{center}
\end{frame}

\section{The Basic Structure of A Document}
\begin{frame}{A basic structure}
	\LaTeX{} enforces structure, Headings, labels and paragraphs are not just
	styled elements but provide meaning to the text. They will be picked up by
	things such as the table of contents, In things such a headers and can be
	directly referenced against labels are provided. for example: \ref{sec:whystray}
\end{frame}

\begin{frame}[fragile]
  \frametitle{The sections}
\begin{minted}{latex}
  \section
  \subsection
  \subsubsection
  \paragraph
\end{minted}
\end{frame}

\section{Formatting first and Once, The Preamble}%
\begin{frame}{How we separate concerns}
	\includegraphics[width=10cm]{Latex document style.png}
\end{frame}
\begin{frame}[fragile]{A real example}

\begin{minted}[linenos]{latex}
\documentclass{article}
\usepackage[left=2.5cm, right=2.5cm]{geometry}
\usepackage[style=apa, backend=biber]{biblatex}
\addbibresource{ccc3.bib}
\usepackage[linktocpage=true,
            colorlinks=true,
            linkcolor=black,
            urlcolor=blue,
            citecolor=black]{hyperref}

\title{Delegation in role based access control}
\author{UP2063130}
\date{}
\end{minted}

\end{frame}

\section{Citations and reference management}%
\begin{frame}{Citation management should not be painful}
How many have had to cite things formally so far? How many have had to format
inline citations by hand into your document, how many times have you had to
change both the bibliography and the citation itself because you missed
something? This is not fun, its actively painful.
\end{frame}
\begin{frame}{Getting citations}
If you are getting citations use \url{https://zotero.org}.
\autocite{ZoteroYourPersonal} Zotero can grab all the metadata it knows about
on

\begin{itemize}
\item Individual papers on journals
		\autocite{barkaFrameworkRolebasedDelegation2000}
  \item News and magazine articles \autocite{104yearoldChicagoWoman2023}
		\item even just random videos and websites \autocite{ZOMBO}
\end{itemize}

all with the click of a button!
\end{frame}

\begin{frame}[fragile]{Using citations}
using citations is a matter of copying in a line of boiler plate and then using
latex commands that actually deal with the formatting of these things.
\begin{minted}{latex}
% in the preamble
\usepackage[style=apa, backend=biber]{biblatex}
\addbibresource{latex.bib}
% ... in document
% for regular referencing
\autocite{CITEKEY}
% for referencing inline
\textcite{CITEKEY}
% referencing other parts of the documents
\ref{label}
% Finally for adding a bibliography
\printbibliography
\end{minted}
Thats about it! \ref{subsec:bib}
\end{frame}

\subsection{Bibliography}
\label{subsec:bib}
\begin{frame}[allowframebreaks]
\frametitle{References}
\printbibliography[heading=none]
\end{frame}

\section{The next steps}
\begin{frame}{there is more to life \LaTeX{}!}
\begin{itemize}
  \item LaTeX is but one example of a markup language
  \item there are lighter versions such as markdown
  \item In a lot of cases you can use tools such as pandoc to convert from
  simpler forms to larger ones
  \item LaTeX as a tool is massive and learning as you go is a must!
\end{itemize}

\end{frame}

\begin{frame}{Where next?}
  \begin{itemize}
	\item make an account on \href{https://overleaf.com}{overleaf}, or install latex on your computer
	\item check out \href{https://ctan.org/}{The comprehensive tex archive
	  network}. It contaisn all of the packages you would want to use (Over 6000
	of them!) as well as very well typeset documentation
	\item the \href{https://tex.stackexchange.com/}{\TeX{} stack exchange is a
	  great place to debug issues}
	\item sites such as \href{https://latex-tutorial.com/}{latex
	  tutorial} \href{https://latex-beamer.com/}{latex beamer}
	and \href{https://www.overleaf.com/learn}{Overleaf's documentation} form a
	back bone of documentation

  \end{itemize}
\end{frame}


\section{We are done!}
\begin{frame}
	\begin{center}
		\framebox{\huge{Any Questions?}}
	\end{center}
\end{frame}



\end{document}
